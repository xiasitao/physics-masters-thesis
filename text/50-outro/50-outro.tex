% !TeX root = scaffold-50.tex
\renewcommand{\imagepath}{../50-outro/img}

\chapter{Conclusion and Outlook}
In this thesis, the FermiQP demonstrator quantum gas microscopy experiment was introduced as a platform for quantum simulations and quantum computing using fermionic lithium. Parameter tuning in Fermi-Hubbard simulations, the working principle of the quantum gates as well as the projected experimental setup of the vacuum chamber, the cooling cycle, imaging, and the lasers were outlined.

The theory of magneto-optical traps was explained, followed by details of the planned implementation of the three-dimensional magneto-optical trap and the gray molasses of the FermiQP demonstrator. The optical setup for producing the trap laser light was explained along with a description of a partial implementation of this setup. The trap geometry as well as the trap beam polarization and intensity were given detailed consideration in preparation of the actual implementation of the three-dimensional magneto-optical trap.

In its last chapter, this thesis covered the coils for generating strong homogeneous fields and magnetic gradients at the trapped atoms' position. The requirements and constraints of the coils with respect to their design, geometry and the generated magnetic fields were elaborated. A Python library developed within the scope of this thesis  for simulating arrangements of coils was presented. Along with magnetic field measurements, this library was then used to characterize the aforementioned coils and propose an optimal configuration.

Due to massive supply chains problems, neither the magneto-optical trap nor the coil arrangement could be put into operation within the writing of this thesis. The next steps for the experimenters of the FermiQP demonstrator include further experimental optimization of the coil arrangement and the assembly of the optics of the three-dimensional magneto-optical trap around the experiment chamber as well as its optimization and  characterization.

Improving and adding more functionality to the coil simulation library would not only make it a more handy tool for planning coils in future quantum gas experiments, it could eventually be turned into a publicly accessible software library for a broader user community.

When put into operation, the three-dimensional magneto-optical trap as well as the coils designed in this thesis will be main components of a machine having the potential to write history as the prototypical ultracold atom-based quantum processor that set new records for scalability, versatility, and stability in the quantum age.