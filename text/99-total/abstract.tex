Quantum gas microscopes have proven their potential as a platform for quantum many-body simulations within the last decade and are traded as a promising candidate for scalable quantum computing. The FermiQP project aims to develop a versatile fermionic lithium-based demonstrator quantum gas microscope combining a quantum simulator and a quantum processor in one device. This thesis is part of the construction phase of this demonstrator experiment. Within its scope, a 3D magneto-optical trap was designed for cooling and capturing lithium atoms. Additionally, magnetic field coils serving as a part of the 3D magneto-optical trap and for tuning atom interactions with Feshbach resonances were developed and characterized.
\vspace{2cm}

Quantengasmikroskope haben im letzten Jahrzehnt ihr Potential als Plattform für Vielteilchen-Quantensimulationen gezeigt und werden als vielversprechende Basis für skalierbare Quantencomputer gehandelt. Das FermiQP-Projekt hat sich zum Ziel gesetzt, ein vielseitig einsetzbares Quantengasmikroskop als Demonstrationsexperiment zu entwickeln, das einen Quantensimulator und einen Quantencomputer kombiniert. Diese Arbeit ist Teil der Aufbauphase dieses Demonstrationsexperiments. Sie dokumentiert einerseits die Planung einer dreidimensionalen magneto-optischen Falle für das Fangen und Kühlen von Lithiumatomen. Andererseits behandelt sie die Entwicklung und Charakterisierung von Magnetfeldspulen als Teil der magneto-optischen Falle und für die Steuerung von Atom-Wechselwirkungen mittels Feshbach-Resonanzen.