Quantum gas microscopes have proven their potential for quantum many-body simulations within the last decade and are traded as a promising candidate platform for scalable quantum computing. The FermiQP project aims to develop a versatile fermionic lithium-based demonstrator quantum gas microscope combining a quantum simulator and a quantum computer in one device. This thesis is part of the construction phase of this demonstrator experiment. Within its scope, a 3D magneto-optical trap was designed for cooling and capturing lithium atoms. Additionally, magnetic field coils serving as a part of this magneto-optical trap and for tuning atom interactions with Feshbach resonances were developed and characterized.
\vspace{2cm}

Quantengasmikroskope haben im letzten Jahrzehnt ihr Potential für Vielteilchen-Quantensimulationen unter Beweis gestellt und werden als vielversprechende Plattform für skalierbare Quantencomputer gehandelt. Das FermiQP-Projekt hat sich zum Ziel gesetzt, ein vielseitig einsetzbares Quantengasmikroskop auf Basis von fermionischem Lithium als Demonstrationsexperiment zu entwickeln, das einen Quantensimulator und einen Quantencomputer kombiniert. Als Teil der Aufbauphase dieses Demonstrationsexperiments dokumentiert diese Arbeit einerseits die Entwicklung einer dreidimensionalen magneto-optischen Falle für das Fangen und Kühlen von Lithiumatomen. Andererseits behandelt sie die Planung und Charakterisierung von Magnetfeldspulen als Teil dieser magneto-optischen Falle und für die Steuerung von Atom-Wechselwirkungen mittels Feshbach-Resonanzen.