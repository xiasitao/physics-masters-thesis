\chapter*{Acknowledgements}

I want to express my gratitude to Immanuel Bloch, Timon Hiker, and Philipp Preiss for giving me the opportunity to take part in launching and constructing this visionary experiment. I am very glad about all the different experiences coming along with setting up a lab from absolute scratch. To me, it was a very fascinating mixture of planning, simulating, organizing, and building, as well as the deep insights into the numerous relevant aspects of physics. I especially appreciated the duality of physics and project development being the leitmotif of every aspect of work on the experiment. Special thanks go to Timon and Philipp for the comprehensive and detailed support in both of these aspects.

\begin{minipage}{\textwidth}
With all my heart I also want to thank Naman, Janet, Simon, Andreas, Robin, Gaurav, Jan (G.), Jan (W.), Yudong, Cady, Sujay, Soumya, Daniel, Jin, Aki, Gleb, Arne, and Alberto for the amazing time I had with you in the lab, in the office, at conferences, and also outside the scope of work. Special thanks go to Janet for the tons of advice she gave me in the lab, the constructive discussions about \href{https://www.spencersmot.co.uk/wp-content/uploads/2021/08/AdobeStock_322578000-1024x819.jpeg}{MOTs} (and the funny moments when finding out about our misconceptions about them), her everlasting constructive janergy, her advice about Italian pronunciation, and for teaching me how to auditorily correctly sip coffee. Also to Robin for spending so much time and patience explaining all sorts of physical phenomena, sharing all his knowledge about how to excel in the lab, for the deep talks about politics, society, and humanity, for the amazing sounds of Latin American music, and for explaining how a FermiPQ really works. To Andreas for all the insights about vacuum, \href{https://i.imgur.com/mamNBfv.jpeg}{CAD} drawings, programming, IT, and for the feedback about this thesis, as well as for the exhausting sessions in the gym, for teaching us how to pronounce \href{https://www.youtube.com/watch?v=_3_JVVs2Kls}{\GaUsS}, and for wiring up the lab with \href{https://soundcloud.com/schatturus/usb-cable-1/s-Fyd3wafUQub?si=9655280d540d4f4b88e9b6aca0bb6602&utm_source=clipboard&utm_medium=text&utm_campaign=social_sharing}{USB cables}. To Simon for the mind-blowing discussions about all sorts of physics, the inspiration from his seemingly unbounded craft and engineering skill set, for founding the Picowatt Gang, goggle dancing sessions, and for being a coil kid who finds my jokes funny.  To Yudong for all the enlightening discussions about optical dipole traps, for reminding me how to pronounce \href{https://soundcloud.com/schatturus/baseplate/s-zBKCu5GdivC?si=674fdd5e94994e409cc93763e37c101e&utm_source=clipboard&utm_medium=text&utm_campaign=social_sharing}{baseplate}, and for establishing what happens \href{https://youtu.be/4G6QDNC4jPs?t=33}{everytime we touch} (red buttons, nano-coatings, or the boundaries of good humor). To Naman for all the inspiring and eye-opening discussions about physics and what is beyond, for the unimaginable passion that his words convey, and for his quite particular kind of humor that is always absolutely enjoyable. To Gaurav for the absolutely inspiring attitude towards what one can accomplish in physics, for the deep and enriching talks, for laughing about my jokes in the most loyal manner possible, as well as for ignoring my definite and inadvertent disability to pronounce his \href{https://upload.wikimedia.org/wikipedia/commons/c/c7/Rothschild%27s_Giraffe_%28Giraffa_camelopardalis_rothschildi%29_male_%287068054987%29%2C_crop_%26_edit.jpg}{name} correctly. To Aki, Jin, and Cady for helping me keep my Mandarin alive. To Jan W. for the feedback about and the extensions for my coil library. To Sujay about the interesting discussions about life in different cultures and countries. And to Daniel for the kicker games and for comforting everyone about expressing frustration with nasty software. I also want to thank everyone else from the group for all the discussions, for telling tips and tricks for the lab, and for lending out all the equipment we couldn't have done without.

\begin{center}
    \small
    \textit{I was trying to come up with a joke about my microscope. Then I realized I had a bad objective.} (\href{https://upjoke.com/microscope-jokes}{upjoke.com})
\end{center}
\end{minipage}




\chapter*{Selbstständigkeitserklärung}
Hiermit erkläre ich, diese Arbeit selbstständig verfasst und keine anderen als die in der Arbeit angegebenen Quellen und Hilfsmittel verwendet zu haben.

\vspace{5cm}
München, den 5. Dezember 2022

\vspace{3cm}
*** Maximilian ***

\ifprintversion
P*****2 \\
8**** München \\
geboren **** \\
Matrikelnummer: ***
\fi
