% !TeX root = scaffold-10.tex
\renewcommand{\imagepath}{../10-intro/img}

\chapter{Introduction}
Digital technology has become an indispensable tool which much of today's knowledge about the world relies on. Data processing, computational modelling, and simulation helped solve many problems in research that could not have been tackled before the silicon age. But yet, digital computing is not able to solve all computational problems efficiently. Especially the exploration of the quantum world poses a serious threat to this success story. While the behavior of quantum systems with only a few constituents can still be predicted with modern computer technology, larger quantum systems will prolong simulation runtimes exponentially due to the exponential growth of dimensionality with their size.

Already in 1982, Richard Feynman came up with a neat \textit{hack} to avoid this disproportional scaling of computational effort, namely not to use classical computers rather than quantum systems for simulation quantum systems~\cite{feynman_simulating_1982}. By mapping the problems in question to well-controllable environments, the quantum system itself takes care of the issue of high dimensionality.

The idea of quantum computing~\cite{hidary_quantum_2021, mainzer_quantencomputer_2020} even goes one step further: The aim of quantum computers is to take advantage of fundamental features of quantum mechanics, most notably superposition and entanglement\todo{More?}, in order to reduce the computational complexity to manageable orders of magnitude\todo{not precise enough?}. They perhaps most famous example of such algorithms was developed by Shor in 1994 for factoring large numbers~\cite{shor_algorithms_1994}.

Quantum simulations and quantum computing 



