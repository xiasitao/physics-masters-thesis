% !TeX root = scaffold-10.tex
\renewcommand{\imagepath}{../10-intro/img}

\chapter{Introduction}
Digital technology has become an indispensable part of people's personal and professional lives in the 21st century. Its ubiquity is a testament to its indisputable success and appears to prove its unlimited capabilities. Also and particularly in science, collecting and processing large amounts of data, computational modelling, and complex simulations have facilitated insights that had been unimaginable before the silicon age. Nevertheless, even the most advanced supercomputers are not able to efficiently solve some of today's most demanding scientific computational problems. Especially simulations of quantum systems pose a serious threat to this success story. While the behavior of a few particles can still be predicted with modern computer technology, for quantum systems with many constituents simulation runtimes increase enormously due to the exponential growth of dimensionality with their size.

Already in 1982, Richard Feynman suggested a \textit{neat hack} to avoid this disproportional scaling of computational effort when simulating quantum systems. Rather than classical computers, he suggested to use, likewise, quantum systems for this task~\cite{feynman_simulating_1982}. By mapping the problems in question to controllable quantum arrangements, they themselves mitigate the issue of high dimensionality.

Not much later, the idea of quantum computing~\cite{nielsen_quantum_2010, hidary_quantum_2021, ladd_quantum_2010, mainzer_quantencomputer_2020} emerged, proposing to utilize quantum systems to make high problem dimensionality manageable for tackling computationally complex questions from outside the quantum realm. Taking advantage of fundamental features of quantum mechanics, most notably superposition and entanglement, it allows certain algorithms to be executed with sub-exponential computational effort, bringing their execution down to reasonable time scales. The perhaps most famous of these algorithms was developed by Peter Shor in 1994 for factoring large numbers~\cite{shor_algorithms_1994}. Over the last two decades, a multitude of different platforms for quantum computing were suggested, ranging from nuclear magnetic resonance~\cite{cory_ensemble_1997}, over photons~\cite{duan_scalable_2004,zhong_quantum_2020} and ions~\cite{home_complete_2009}, to superconducting circuits~\cite{dicarlo_demonstration_2009, arute_quantum_2019-1}, and quantum dots~\cite{hanson_spins_2007}.

Another alternative are gases of ultracold neutral atoms~\cite{deutsch_quantum_2000}. By trapping them in regular patterns with optical lattices or optical tweezers, they can be used as a well-controllable basis for quantum simulations and quantum computing, standing out due to their long coherence times and high scalability. While these highly-ordered systems of atoms have already produced remarkable results as quantum simulators for bosonic and fermionic quantum materials~\cite{bloch_quantum_2012, gross_quantum_2017}, developments towards implementing them for universal gate-based quantum computing are still at the beginning.

In 2021, the FermiQP project was launched with the goal to combine both a quantum simulator and a quantum computer in a single quantum gas microscopy experiment, aiming to surpass existing quantum computers in terms of the number of qubits.

For efficient and precise operation, the experiment needs, among other things, a compact design, a fast cooling cycle, and stable and homogeneous magnetic fields. As a part of the construction phase of the experiment, this thesis project contributed to the plannings and the implementation of these features. In particular, a three-dimensional magneto-optical trap as a main component of the experiment's cooling cycle was developed. In addition, magnetic field coils to be used for this trap and for atom interaction tuning with Feshbach resonances were designed and characterized. The next chapter gives a more thorough overview of the FermiQP demonstrator experiment, outlining how cooling, imaging, interaction tuning, and quantum gates will be implemented. The third chapter then covers the theory of magneto-optical traps and reports on the development of the three-dimensional magneto-optical trap for the FermiQP demonstrator experiment. The following chapter then discusses the specification of the magnetic field coils and characterizes the homogeneous magnetic fields and magnetic gradients they generate. The last chapter offers a conclusion and an outlook on the next steps towards integrating these two components into the experiment.


