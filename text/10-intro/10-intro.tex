% !TeX root = scaffold-10.tex
\renewcommand{\imagepath}{../10-intro/img}

\chapter{Introduction}
The ubiquity of digital technology in the 21st century appears to prove its unlimited capabilities. Having become an indispensable part of people's personal and professional lives seems to mark its indisputable success. Also and particularly in science, collecting and processing large amounts of data, computational modelling, and complex simulations have facilitated insights that were unimaginable before the silicon age. But yet, even the most advanced supercomputers are not able to solve some of today's most demanding scientific computational problems efficiently. Especially simulations of quantum systems pose a serious threat to this success story. While the behavior of only a few particles can still be predicted with modern computer technology, for quantum systems with many constituents simulation runtimes will increase enormously due to the exponential growth of dimensionality with their size.

Already in 1982, Richard Feynman suggested a neat hack to avoid this disproportional scaling of computational effort when simulating quantum systems. He suggested not to use classical computers for this task rather than, likewise, quantum systems~\cite{feynman_simulating_1982}. By mapping the problems in question to controllable quantum arrangements, they themselves mitigate the issue of high dimensionality.

Not much later, the idea of quantum computing~\cite{nielsen_quantum_2010, hidary_quantum_2021, ladd_quantum_2010, mainzer_quantencomputer_2020} emerged, proposing to utilize this way of making high problem dimensionality manageable for tackling computationally complex questions from outside the quantum realm. Taking advantage of fundamental features of quantum mechanics, most notably superposition and entanglement, it allows certain algorithms to be executed with sub-exponential computational effort, bringing their execution down to reasonable time scales. The perhaps most famous of such algorithms was developed by Shor in 1994 for factoring large numbers~\cite{shor_algorithms_1994}. Over the last two decades, a multitude of different platforms for quantum computing were suggested, ranging from nuclear magnetic resonance~\cite{cory_ensemble_1997}, over photons~\cite{duan_scalable_2004,zhong_quantum_2020} and ions~\cite{home_complete_2009}, to superconducting circuits~\cite{dicarlo_demonstration_2009, arute_quantum_2019-1} and quantum dots~\cite{hanson_spins_2007}.

Another alternative are gases of ultracold neutral atoms. By trapping them in regular patterns with optical lattices or optical tweezers, they can be used as a well-controllable basis for quantum simulations and quantum computing, promising long coherence times and high scalability. While these highly-ordered systems of atoms have already produced remarkable results as quantum simulators for bosonic and fermionic quantum materials~\cite{bloch_quantum_2012, gross_quantum_2017}, developments towards implementing them for universal gate-based quantum computing are still at the beginning.

In 2021, the FermiQP project was launched with the goal to combine both a quantum simulator and a quantum computer in a single ultracold quantum gas microscopy experiment, aiming to set new standards in terms of precision and the number of qubits. 

This thesis is a part of the build-up phase of this demonstrator experiment. Within its scope, a three-dimensional magneto-optical trap as a main component of the experiment's cooling cycle along with magnetic field coils for atom interaction tuning with Feshbach resonances were designed. The next chapter gives a more thorough overview of the FermiQP demonstrator experiment, outlining how cooling, imaging, interaction tuning, and quantum gates will be implemented. The third chapter then covers the theory of magneto-optical traps and reports on the development of the three-dimensional magneto-optical trap for the FermiQP demonstrator experiment. The fourth chapter then discusses the specification and characterization of the magnetic field coils to be used for the three-dimensional magneto-optical trap and for generating high homogeneous magnetic fields. The last chapter concludes with an outlook on the next steps towards integrating these two components into the experiment.


