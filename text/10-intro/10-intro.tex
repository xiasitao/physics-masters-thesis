% !TeX root = scaffold-10.tex
\renewcommand{\imagepath}{../10-intro/img}

\chapter{Introduction}
Digital technology has become an indispensable tool which much of today's knowledge about the world relies on. Data processing, computational modelling, and simulation facilitated solving many problems in research that could not have been tackled before the silicon age. But yet, even the most advanced computer are not able to solve all of today's  most computational problems efficiently. Especially simulations of the quantum world pose a serious threat to this success story. While the behavior of quantum systems with only a few constituents can still be predicted with modern computer technology, larger quantum systems will prolong simulation runtimes exponentially due to the exponential growth of dimensionality with their size.\todo{shorter?}

Already in 1982, Richard Feynman came up with a neat \textit{hack} to avoid this disproportional scaling of computational effort, namely not to use classical computers rather than quantum systems themselves to simulate quantum systems~\cite{feynman_simulating_1982}. By mapping the problems in question to well-controllable environments, they take care of the issue of high dimensionality themselves.

The idea of quantum computing~\cite{nielsen_quantum_2010, hidary_quantum_2021, mainzer_quantencomputer_2020} even goes one step further: The aim of quantum computers is to take advantage of fundamental features of quantum mechanics, most notably superposition and entanglement\todo{More?}, for reducing the computational complexity in solving certain problems. Special algorithms are designed to run in sub-exponential time on these quantum computers, the perhaps most famous of which was developed by Shor in 1994 for factoring large numbers~\cite{shor_algorithms_1994}. Over the last two decades, a multitude of different platforms for quantum computing and simulation were proposed, ranging from nuclear magnetic resonance, over photon circuits and superconducting circuits, ions\todo{add references}. 

Another particularly promising approach for quantum simulations and quantum computing are gases of neutral atoms which allow long coherence times and better scaling compared to most other platforms. While highly-ordered systems of ultracold bosonic and fermionic atoms in optical lattices and tweezers have already produced astounding results as quantum simulators\todo{references}, developments towards implementing them for universal gate-based quantum computing are still in their beginning.

The FermiQP project aims to combine both quantum simulation and quantum computation in one ultracold atom experiment, setting new records in terms of precision and number of qubits. It was started in 2021 and its demonstrator experiment is still in the construction phase.

This thesis is part of the experiment build-up of this demonstrator of FermiQP. Within its scope, a three-dimensional magento-optical trap as a main component of the experiment's cooling cycle along with magnetic field coils for atom interaction tuning with Feshbach resonances was developed. The rest of it is organized as follows: The next chapter gives a more thorough overview of the FermiQP demonstrator experiment. The third chapter then covers the development of the three-dimensional magneto-optical trap. The fourth chapter discusses the specification and characterization of the magnetic field coils to be use in the magneto-optical trap and for generating Feshbach resonances. The last chapter concludes with an outlook of the next steps towards integration of these two components into the experiment.


