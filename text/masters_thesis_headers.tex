\documentclass[
	twoside=false,
	paper=a4,
	fontsize=11pt,
	parskip=half,
	ngerman
]{scrbook}

\newif\ifprintversion
\printversionfalse

\newif\ifdraft
\drafttrue

% === Page  ===
\ifprintversion
  \usepackage[paper=a4paper,bindingoffset=10mm,twoside]{geometry}
\else 
  \usepackage[paper=a4paper]{geometry}
\fi

\ifdraft
  \usepackage{draftwatermark}
  \SetWatermarkLightness{0.8}
  \SetWatermarkText{preliminary (\today)}
  \SetWatermarkAngle{0}
  \SetWatermarkVerCenter{0.93\pdfpageheight}
  \SetWatermarkScale{0.17}
\fi

% === Fonts  ===
\DeclareUnicodeCharacter{2009}{\,} 
\usepackage{noto}
\usepackage{ebgaramond}
\usepackage[scale=0.75]{sourcecodepro}
\usepackage[cmintegrals,cmbraces]{newtxmath}
\usepackage{ebgaramond-maths}
% Redefining missing symbols
% https://tex.stackexchange.com/questions/215270/can-someone-explain-this-weird-font-behavior-ebgaramond-maths
\makeatletter
  \DeclareSymbolFont{ntxletters}{OML}{ntxmi}{m}{it}
  \SetSymbolFont{ntxletters}{bold}{OML}{ntxmi}{b}{it}
  \re@DeclareMathSymbol{\leftharpoonup}{\mathrel}{ntxletters}{"28}
  \re@DeclareMathSymbol{\leftharpoondown}{\mathrel}{ntxletters}{"29}
  \re@DeclareMathSymbol{\rightharpoonup}{\mathrel}{ntxletters}{"2A}
  \re@DeclareMathSymbol{\rightharpoondown}{\mathrel}{ntxletters}{"2B}
  \re@DeclareMathSymbol{\triangleleft}{\mathbin}{ntxletters}{"2F}
  \re@DeclareMathSymbol{\triangleright}{\mathbin}{ntxletters}{"2E}
  \re@DeclareMathSymbol{\partial}{\mathord}{ntxletters}{"40}
  \re@DeclareMathSymbol{\flat}{\mathord}{ntxletters}{"5B}
  \re@DeclareMathSymbol{\natural}{\mathord}{ntxletters}{"5C}
  \re@DeclareMathSymbol{\star}{\mathbin}{ntxletters}{"3F}
  \re@DeclareMathSymbol{\smile}{\mathrel}{ntxletters}{"5E}
  \re@DeclareMathSymbol{\frown}{\mathrel}{ntxletters}{"5F}
  \re@DeclareMathSymbol{\sharp}{\mathord}{ntxletters}{"5D}
  \re@DeclareMathAccent{\vec}{\mathord}{ntxletters}{"7E}
\makeatother
\renewcommand{\epsilon}{\varepsilon}
\usepackage[]{todonotes}


%\usepackage[osf, sc]{mathpazo}
%\linespread{1.03}
\usepackage{setspace}

\usepackage[T1]{fontenc}
\usepackage{eurosym}
\newcommand{\red}[1]{\colorbox{red}{\textcolor{white}{\textbf{#1}}}}

\addtokomafont{paragraph}{\small}
\addtokomafont{subparagraph}{\tiny}

\makeatletter
\DeclareRobustCommand{\GaUsS}{%
  G\kern-.35em
  {
    \sbox\z@ U\vbox to\ht\z@{
      \hbox{\check@mathfonts\fontsize\sf@size\z@\math@fontsfalse\selectfont A}
      \vss
    }}\kern-.15emU\kern-.1667em\lower.5ex\hbox{S}\kern-.125emS
}
\makeatother


% === Math stuff ===
\usepackage{amsmath}
\usepackage{amsfonts}
\usepackage{mathtools}
\usepackage{braket}
\usepackage[retain-explicit-plus, separate-uncertainty=true]{siunitx}
\DeclareSIUnit\gauss{G}
\sisetup{per-mode=fraction,fraction-function=\tfrac}
\usepackage{physics}
% Preventing italic variables in ket to interfere with the Ket
\newcommand{\KetSpaced}[1]{\Ket{~#1}}
\newcommand{\KetText}[1]{\Ket{\text{#1}}}

% === Tables ===
\usepackage{booktabs}
\usepackage{tabu}
\usepackage{tabularx}
\usepackage{ltablex}
\usepackage{multirow}
\usepackage{longtable}

% === Quotes  ===
\usepackage{csquotes}
\usepackage{epigraph}
\usepackage{listings}
\usepackage{xcolor}
\definecolor{stringcolor}{RGB}{154,91,145}  % rebecca
\definecolor{keywordcolor}{RGB}{80,189,233} % blue
\definecolor{commentcolor}{RGB}{255,0,0}  % red
\lstset{
  basicstyle=\ttfamily,
  numbers=left,
  breaklines=true,
  language=Python,
  otherkeywords={True,False},
  commentstyle=\color{commentcolor},
  keywordstyle=\color{keywordcolor}\bfseries,
  stringstyle=\color{stringcolor},
  emph={self, csl, ConeShapedCoilArrangement, FieldMap, Collection, Jones_vector},
  emphstyle={\color{keywordcolor}},
}

% === Graphics ===
\usepackage{tikz}
\usepackage{subcaption}
\usepackage{pgf}
\usepackage{pgfplots}
\pgfplotsset{compat=1.11}
% \usepackage{float}
\newcommand{\imagepath}{}
\usepackage{rotating}

\definecolor{thesisred}{RGB}{255,0,0}
\definecolor{thesisgray}{RGB}{71,72,71}
\definecolor{thesisrebecca}{RGB}{105,62,163}
\definecolor{thesisfuchsia}{RGB}{154,91,145}
\definecolor{thesisblue}{RGB}{80,189,233}

% === Citations ===
\usepackage[
  %style=numeric-comp,
  style=nature,
  backend=biber,
  sorting=none
]{biblatex}
\addbibresource{../FermiQP.bib}
\PassOptionsToPackage{hyphens}{url}
\AtEveryBibitem{
  \clearfield{urlyear}
}{}

% === Title ===
\title{Developing a 3D Magneto-Optical Trap and Feshbach Fields for a Fermion Quantum Processor}
\newcommand{\titleDE}{Entwicklung einer 3D- magneto-optischen Falle und Planung von Feshbach-Feldern für einen Fermionen-Quantenprozessor}
\author{Tobias Maximilian Philipp Schattauer}
\date{December 2022}

\makeatletter
\usepackage[
	pdftitle={\@title},
	pdfauthor={Tobias Maximilian Philipp Schattauer},
	pdfkeywords={magneto-optical trap, Feshbach coils, quantum gas microscope, master's thesis, FermiQP}
	]{hyperref}
\makeatother