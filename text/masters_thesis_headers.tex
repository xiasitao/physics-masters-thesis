\documentclass[
	twoside=false,
	paper=a4,
	fontsize=11pt,
	parskip=half,
	ngerman
]{scrbook}

\newif\ifprintversion
\printversionfalse

% === Page  ===
\ifprintversion
  \usepackage[paper=a4paper,bindingoffset=10mm,twoside]{geometry}
\else 
  \usepackage[paper=a4paper]{geometry}
\fi

% === Fonts  ===
\usepackage{noto}
\usepackage{ebgaramond}
\usepackage[scale=0.75]{sourcecodepro}
\usepackage[cmintegrals,cmbraces]{newtxmath}
\usepackage{ebgaramond-maths}
% Redefining missing symbols
% https://tex.stackexchange.com/questions/215270/can-someone-explain-this-weird-font-behavior-ebgaramond-maths
\makeatletter
  \DeclareSymbolFont{ntxletters}{OML}{ntxmi}{m}{it}
  \SetSymbolFont{ntxletters}{bold}{OML}{ntxmi}{b}{it}
  \re@DeclareMathSymbol{\leftharpoonup}{\mathrel}{ntxletters}{"28}
  \re@DeclareMathSymbol{\leftharpoondown}{\mathrel}{ntxletters}{"29}
  \re@DeclareMathSymbol{\rightharpoonup}{\mathrel}{ntxletters}{"2A}
  \re@DeclareMathSymbol{\rightharpoondown}{\mathrel}{ntxletters}{"2B}
  \re@DeclareMathSymbol{\triangleleft}{\mathbin}{ntxletters}{"2F}
  \re@DeclareMathSymbol{\triangleright}{\mathbin}{ntxletters}{"2E}
  \re@DeclareMathSymbol{\partial}{\mathord}{ntxletters}{"40}
  \re@DeclareMathSymbol{\flat}{\mathord}{ntxletters}{"5B}
  \re@DeclareMathSymbol{\natural}{\mathord}{ntxletters}{"5C}
  \re@DeclareMathSymbol{\star}{\mathbin}{ntxletters}{"3F}
  \re@DeclareMathSymbol{\smile}{\mathrel}{ntxletters}{"5E}
  \re@DeclareMathSymbol{\frown}{\mathrel}{ntxletters}{"5F}
  \re@DeclareMathSymbol{\sharp}{\mathord}{ntxletters}{"5D}
  \re@DeclareMathAccent{\vec}{\mathord}{ntxletters}{"7E}
\makeatother
\renewcommand{\epsilon}{\varepsilon}
\usepackage[]{todonotes}


%\usepackage[osf, sc]{mathpazo}
%\linespread{1.03}
\usepackage{setspace}

\usepackage[T1]{fontenc}
\usepackage{eurosym}
\newcommand{\red}[1]{\colorbox{red}{\textcolor{white}{\textbf{#1}}}}

\addtokomafont{paragraph}{\small}
\addtokomafont{subparagraph}{\tiny}


% === Math stuff ===
\usepackage{amsmath}
\usepackage{amsfonts}
\usepackage{mathtools}
\usepackage{braket}
\usepackage[retain-explicit-plus, separate-uncertainty=true]{siunitx}
\DeclareSIUnit\gauss{G}
\sisetup{per-mode=fraction,fraction-function=\tfrac}
\usepackage{physics}
% Preventing italic variables in ket to interfere with the Ket
\newcommand{\KetSpaced}[1]{\Ket{~#1}}
\newcommand{\KetText}[1]{\Ket{\text{#1}}}

% === Tables ===
\usepackage{booktabs}
\usepackage{tabu}
\usepackage{tabularx}
\usepackage{ltablex}
\usepackage{multirow}
\usepackage{longtable}

% === Quotes  ===
\usepackage{csquotes}
\usepackage{epigraph}
\usepackage{listings}
\lstset{
  basicstyle=\ttfamily,
  numbers=left
}

% === Graphics ===
\usepackage{tikz}
\usepackage{subcaption}
\usepackage{pgf}
% \usepackage{float}
\newcommand{\imagepath}{}

% === Citations ===
\usepackage[
  %style=numeric-comp,
  style=nature,
  backend=biber,
  sorting=none
]{biblatex}
\addbibresource{../FermiQP.bib}
\PassOptionsToPackage{hyphens}{url}
\AtEveryBibitem{
  \clearfield{urlyear}
}{}

% === Title ===
\usepackage[
	pdftitle={TODO MASTERS THESIS},
	pdfauthor={Tobias Maximilian Philipp Schattauer},
	pdfkeywords={Magneto-optical trap, Feshbach coils, Quantum gas microscope, Master's thesis, FermiQP}
	]{hyperref}

\title{Designing a 3D Magneto-optical Trap and Feshbach Fields for a Fermion Quantum Processor}
\newcommand{\titleDE}{Entwicklung einer 3D-magneto-optischen Falle und von Feshbach-Feldern für einen fermionischen Quantenprozessor}
\author{Tobias Maximilian Philipp Schattauer}
\date{December 2022}